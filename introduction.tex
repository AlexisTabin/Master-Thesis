\section{Introduction}
\label{sec:introduction}

Healthcare costs are globally increasing due to an aging population, technological advancements, medication errors, and a rise in annual spending on medicines  \cite{fries_james_f_reducing_1993, bodenheimer_high_2005}. The aging population, poor diet, physical inactivity, and tobacco use (including secondhand smoke) \cite{cdc_chronic_2022} contribute to the prevalence of chronic diseases, which are a major cause of deaths among the populations \cite{huzooree_pervasive_2019}. Digital health \cite{varshney_mobile_2014} can fulfill the need for healthcare accessible to everyone, regardless of location or time, while also improving the quality of care and reducing costs. The use of portable edge devices with sensing capabilities allows for the remote monitoring of patient health data, which can be particularly helpful for those with chronic conditions \cite{javaid_sensors_2021}. With wearable sensors, mobile phones, or other edge devices, patients can easily record physiological and behavioral data, which can be aggregated to create digital biomarkers that explain, influence, or predict health-related outcomes. Passively measured data such as vital signs, physical activity, and other health-related data allows patients to monitor their health condition without visiting a healthcare provider~\cite{coravos_developing_2019}. These real-time and remote monitoring capabilities not only improve patient outcomes but can also reduce healthcare costs by minimizing the need for frequent visits with clinicians.

Processing huge amounts of sequence data typically requires versatile, high-performing, and highly generalizing \gls{dl} models.
 However, most existing studies have concentrated on tertiary prevention~\cite{barata_bitemporal_2024}, which aims to prevent disease recurrence or complications. Tertiary prevention only deals with diseases that have already occurred and does not proactively reduce the burden on healthcare systems. Therefore, it is crucial to shift the focus towards the early detection of diseases (secondary prevention) or even preventing diseases from occurring in the first place (primary prevention)~\cite{vlachopoulos_role_2015}.
Both primary and secondary prevention can be very beneficial in preventing the onset of serious health concerns. Research in this field has been limited due to uncertainty about which factors to examine. In general, it is difficult to evaluate the overall health condition of a healthy individual in the absence of disease symptoms.
% We do not yet have a method to assess the general health state of a healthy individual.
One potential method to determine the general health state of a person is by using the concept of biological age. Prior work has shown that it is possible to use deep learning models to predict a person's biological age non-invasively using physical activity data~\cite{pyrkov_extracting_2018, rahman_deep_2019}.
Nevertheless, existing methods of predicting biological age lack an explanation about what a person can do to improve their health state in general. It is widely known that general recommendations such as taking a minimum number of steps each day~\cite{tudor-locke_how_2011}, maintaining good sleeping habits~\cite{shim_wearable-based_2023}, and engaging in recreational activities~\cite{saxena_mental_2005} have a significant impact on health outcomes. However, specific recommendations tailored to the individual's needs are currently lacking, making it difficult to identify what changes to make at a personal level. In order to provide these recommendations, the individual needs to understand how the model assesses their health status.

Despite achieving great performance, \gls{ai} models are limited due to being seen as a black box, resulting in low practical use, especially in healthcare. \gls{xai} helps developers, domain experts, and users understand how \gls{dl} models work and how they make predictions \cite{loh_application_2022}. Many state-of-the-art tools for explaining \gls{dl} models rely on visually highlighting important input data areas, which is useful for developers or domain experts but hard for patients to understand. Counterfactual explanation systems \cite{byrne_counterfactuals_2019} aim to support counterfactual reasoning by modifying the input data to lead to a different prediction by the model. That way, the users of counterfactual explanation systems are provided with a fully diverse type of illustrative information that complies with the \gls{gdpr}\cite{wachter_counterfactual_2018} and are easy for humans to understand \cite{miller_explanation_2019}. There is a tremendous potential for counterfactual explanations in the mobile health setting \cite{lee_clinical_2024}. 

Yet, many of the \gls{xai} techniques predominantly deal with images or texts; time series data has attracted less interest, and the few techniques developed for time series are focused on tasks such as classification or forecasting \cite{theissler_explainable_2022}.     Especially in medical contexts, where relevant information often consists of time-dependent information, high-quality time series counterfactuals have the potential to give meaningful insights into decision processes.

With our work, we make the following contributions:
\begin{itemize}
    \item We present a novel approach for generating counterfactual explanations for time series extrinsic regression.
    \item We use our approach to adjust four counterfactual methods for time series classification to time series extrinsic regression.
    \item We compare qualitatively and quantitatively generated counterfactual explanations in a mobile health setting to estimate biological age from physical activity data.
    \item We illustrate how counterfactual explanations can be used to generate meaningful text recommendations and provide continuous health supervision, thus reducing the need for external supervision and, consequently, healthcare costs.
\end{itemize}

\begin{table}[h!t]
\caption{Code Availability}
\label{tab:introduction:code_and_website}
\centering
\renewcommand*{\arraystretch}{1.4}
\begin{tabularx}{\columnwidth}{l|X}
\hline
\textbf{Implementation} & Python, R\\
\textbf{License} & MIT \\
% \textbf{Documentation} & \url{https://www.claid.ethz.ch}\\
% \textbf{Available packages} & pip, pub, aar (maven)\\
\textbf{Code repository} & \url{https://github.com/RealLast/BA-Estimation-TCN}\\
\hline
\end{tabularx}
\end{table}

%============ Written by ChatGPT ============
