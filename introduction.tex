\section{Introduction}
\label{sec:introduction}

% \textbf{Health related aspects}
% Healthcare costs are exploding
% Monitoring health variables on a regular basis can help prevent chronic diseases rather than just reacting to them after they have developed.

% \textbf{Deep Learning can infer Biological Age from Physical Activity}
% AI can be used to monitor health simply through PA
% But AI cannot tell how to change.

% \textbf{Goal of the paper}
% Explain what the patient should do to improve his health. 
% Show that there is a gap in the research and that it is very much needed in healthcare.


%============ Written by ChatGPT ============
In contemporary healthcare landscapes, the challenge of escalating healthcare costs looms large. Amidst this backdrop, the imperative of shifting from reactive to proactive healthcare strategies gains prominence. One promising avenue is the utilization of deep learning methodologies to glean insights from an individual's \acrfull{pa} data, thereby inferring crucial health indicators such as biological age. However, while the potential of \acrfull{ai} in health monitoring is vast, a critical gap persists: the translation of \acrshort{ai}-derived insights into actionable steps for individuals to improve their health.

\textbf{Health Related Aspects} Healthcare costs continue to spiral upward, underscoring the urgency of preventative healthcare measures. Monitoring health variables regularly presents a proactive approach to mitigating the onset of chronic diseases, contrasting with the traditional reactive model of healthcare intervention. Moreover, the burden of chronic diseases on both individuals and healthcare systems necessitates a paradigm shift towards proactive measures aimed at early detection and prevention.

\textbf{Deep Learning Can Infer Biological Age from Physical Activity} The advent of deep learning algorithms has facilitated the extraction of nuanced health information from \acrshort{pa} data alone. Yet, while \acrshort{ai} can proficiently analyze and infer biological age, it currently falls short in providing personalized guidance on how individuals can effectuate tangible changes in their lifestyle to enhance their health outcomes.

\textbf{Goal of the Paper} This paper endeavours to bridge this crucial gap by elucidating actionable steps individuals can undertake to ameliorate their health trajectory based on insights derived from \acrshort{ai}-driven analysis of their physical activity data. By demonstrating the dearth of research addressing this vital translation step, we underscore the pressing need for such interventions within the healthcare domain.

In the ensuing sections, we delve into the methodologies employed for inferring biological age from physical activity data and subsequently propose a framework for translating these insights into personalized health recommendations. Through this exploration, we aim to underscore the transformative potential of integrating \acrshort{ai}-driven health monitoring with actionable guidance, thereby revolutionizing preventative healthcare practices.
%============ Written by ChatGPT ============
