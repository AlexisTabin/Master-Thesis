Noninvasive, smart, and wearable devices such as smartphones and smartwatches collect a vast amount of multimodal data that potentially contains valuable information about our daily activities and habits.
% 
Machine learning methods can aggregate these data to create digital biomarkers. These digital biomarkers can potentially enable personalized and timely interventions to detect and prevent chronic diseases at an early stage. However, current digital biomarkers cannot prevent diseases before they occur because they are currently unable to assess a person's general health state. 
% 
By leveraging advancements in digital health, we investigate the utility of generating recommendations to improve a person's general health state based on physical activity data. We use deep learning models to predict biological age as a proxy for a person's general health state from physical activity data.
% 
To generate recommendations using the deep learning models, we formally introduce four novel counterfactual algorithms for time series extrinsic regression tasks derived from prior work on time series classification. Furthermore, we conducted qualitative and quantitative analyses of their performances and obtained indicative health recommendations that patients could use to improve their daily habits.