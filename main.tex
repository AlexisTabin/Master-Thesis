%% 
%% Copyright 2007-2020 Elsevier Ltd
%% 
%% This file is part of the 'Elsarticle Bundle'.
%% ---------------------------------------------
%% 
%% It may be distributed under the conditions of the LaTeX Project Public
%% License, either version 1.2 of this license or (at your option) any
%% later version.  The latest version of this license is in
%%    http://www.latex-project.org/lppl.txt
%% and version 1.2 or later is part of all distributions of LaTeX
%% version 1999/12/01 or later.
%% 
%% The list of all files belonging to the 'Elsarticle Bundle' is
%% given in the file `manifest.txt'.
%% 
%% Template article for Elsevier's document class `elsarticle'
%% with harvard style bibliographic references

\documentclass[5p,times]{elsarticle}
%% Use the option review to obtain double line spacing
%% \documentclass[authoryear,preprint,review,12pt]{elsarticle}

%% Use the options 1p,twocolumn; 3p; 3p,twocolumn; 5p; or 5p,twocolumn
%% for a journal layout:
%% \documentclass[final,1p,times,authoryear]{elsarticle}
%% \documentclass[final,1p,times,twocolumn,authoryear]{elsarticle}
%% \documentclass[final,3p,times,authoryear]{elsarticle}
%% \documentclass[final,3p,times,twocolumn,authoryear]{elsarticle}
%% \documentclass[final,5p,times,authoryear]{elsarticle}
%% \documentclass[final,5p,times,twocolumn,authoryear]{elsarticle}

%% For including figures, graphicx.sty has been loaded in
%% elsarticle.cls. If you prefer to use the old commands
%% please give \usepackage{epsfig}

%% The amssymb package provides various useful mathematical symbols
\usepackage{amssymb}
%% The amsthm package provides extended theorem environments
%% \usepackage{amsthm}

%% The lineno packages adds line numbers. Start line numbering with
%% \begin{linenumbers}, end it with \end{linenumbers}. Or switch it on
%% for the whole article with \linenumbers.
%% \usepackage{lineno}

\usepackage[colorlinks=true, allcolors=blue]{hyperref}
\usepackage{booktabs,chemformula}
\usepackage{bbm}
\usepackage{algpseudocode}
\usepackage{amsmath}
\usepackage{optidef}
\usepackage{tablefootnote}

\usepackage{hyperref}
\usepackage[acronym]{glossaries}
\usepackage{multirow}
\usepackage{subcaption}
\usepackage{lipsum}
\hypersetup{%
  colorlinks=true,
  linkcolor=bluegray,
  citecolor=black,
  urlcolor=black,
  linktoc=all,
}

\renewcommand*{\glstextformat}[1]{\textcolor{black}{#1}}

\usepackage{makecell}
\usepackage{booktabs}
\usepackage{tabularx}
\usepackage{pdfpages}
\usepackage{lineno}

% https://people.inf.ethz.ch/markusp/teaching/guides/guide-tables.pdf
\renewcommand{\arraystretch}{1.2}


\makeglossaries
\loadglsentries{glossary}


% \theoremstyle{definition}
\newtheorem{definition}{Definition}[section]

\journal{ETH Zurich}

\begin{document}

\newcommand{\valid}[2]{f(#1) - \varepsilon \geq f(#2) \geq f(#1) - 2\varepsilon}
% \includepdf[pages=-]{Declaration_of_originality.pdf}
% \includepdf[pages={1}]{myfile.pdf
\begin{frontmatter}

%% Title, authors and addresses

%% use the tnoteref command within \title for footnotes;
%% use the tnotetext command for theassociated footnote;
%% use the fnref command within \author or \affiliation for footnotes;
%% use the fntext command for theassociated footnote;
%% use the corref command within \author for corresponding author footnotes;
%% use the cortext command for theassociated footnote;
%% use the ead command for the email address,
%% and the form \ead[url] for the home page:
%% \title{Title\tnoteref{label1}}
%% \tnotetext[label1]{}
%% \author{Name\corref{cor1}\fnref{label2}}
%% \ead{email address}
%% \ead[url]{home page}
%% \fntext[label2]{}
%% \cortext[cor1]{}
%% \affiliation{organization={},
%%            addressline={}, 
%%            city={},
%%            postcode={}, 
%%            state={},
%%            country={}}
%% \fntext[label3]{}


% \title{Towards Explainable Time-Series Extrinsic Regression for Biological Age Estimation}
\title{How to improve your health by changing your habits:\\
Towards Explainable Deep Learning Extrinsic Regression for Biological Age Estimation
}

%% use optional labels to link authors explicitly to addresses:
%% \author[label1,label2]{}
%% \affiliation[label1]{organization={},
%%             addressline={},
%%             city={},
%%             postcode={},
%%             state={},
%%             country={}}
%%
%% \affiliation[label2]{organization={},
%%             addressline={},
%%             city={},
%%             postcode={},
%%             state={},
%%             country={}}
% Julia E. Vogt
% View ORCID ID profile
% Department of Computer Science, ETH Zurich, Universitätstrasse 6, 8092, Zürich, Switzerland

\author[inst2]{Alexis Tabin}
\author[inst1]{Patrick Langer}
\author[inst2]{Julia E. Vogt}
\author[inst1]{Filipe Barata}

\affiliation[inst2]{organization={ETH Zurich},%Department and Organization
            addressline={Department of Computer Science}, 
            city={Zurich},
            country={Switzerland}}
\affiliation[inst1]{organization={ETH Zurich},
            addressline={ Centre for Digital Health Interventions}, 
            city={Zurich},
            country={Switzerland}}
\begin{abstract}
% Pyrkov
% Age-related physiological changes in humans are linearly associated with age. Naturally, linear combinations of physiological measures trained to estimate chronological age have recently emerged as a practical way to quantify aging in the form of biological age. In this work, we used one-week long physical activity records from a 2003–2006 National Health and Nutrition Examination Survey (NHANES) to compare three increasingly accurate biological age models: the unsupervised Principal Components Analysis (PCA) score, a multivariate linear regression, and a state-of-the-art deep convolutional neural network (CNN). We found that the supervised approaches produce better chronological age estimations at the expense of a loss of the association between the aging acceleration and all-cause mortality. Consequently, we turned to the NHANES death register directly and introduced a novel way to train parametric proportional hazards models suitable for out-of-the-box implementation with any modern machine learning software. As a demonstration, we produced a separate deep CNN for mortality risks prediction that outperformed any of the biological age or a simple linear proportional hazards model. Altogether, our findings demonstrate the emerging potential of combined wearable sensors and deep learning technologies for applications involving continuous health risk monitoring and real-time feedback to patients and care providers. \\

% %% Syed Ashiqur Rahman
% Recent research highlights the need for a correct instrument for monitoring individual health status, especially in the elderly. Different definitions of biological ageing have been proposed, with a consistent positive association of physical activity and physical fitness with decelerated ageing trajectories. The six-minute walking test is considered the current gold standard for estimating the individual fitness status of the elderly.

% Human age estimation is an important and difficult challenge. Different biomarkers and numerous approaches have been studied for biological age estimation, each with its advantages and limitations. In this work, we investigate whether physical activity can be exploited for biological age estimation for adult humans. We introduce an approach based on deep convolutional long short term memory (ConvLSTM) to predict biological age, using human physical activity as recorded by a wearable device.  the NHANES physical activity dataset. This work has significant implications in combining wearable sensors and deep learning techniques for improved health monitoring, for instance, in a mobile health environment. Mobile health (mHealth) applications provide patients, caregivers, and administrators continuous information about a patient, even outside the hospital.
% \\

% %Jinjoo Shim
% Repeated disruptions in circadian rhythms are associated with implications for health outcomes and longevity. The utilization of wearable devices in quantifying circadian rhythm to elucidate its connection to longevity, through continuously collected data remains largely unstudied. In this work, we investigate a data-driven segmentation of the 24-h accelerometer activity profiles from wearables as a novel digital biomarker for longevity in 7,297 U.S. adults from the 2011–2014 National Health and Nutrition Examination Survey. Using hierarchical clustering, we identified five clusters and described them as follows: \textit{High activity}, \textit{Low activity}, \textit{Mild circadian rhythm (CR) disruption”}, \textit{Severe CR disruption}, and \textit{Very low activity}.
% Young adults with extreme CR disturbance are seemingly healthy with few comorbid conditions, but in fact associated with higher white blood cell, neutrophils, and lymphocyte counts  Older adults with CR disruption are significantly associated with increased systemic inflammation indexes , biological aging advance , and all-cause mortality risk. Our findings highlight the importance of circadian alignment on longevity across all ages and suggest that data from wearable accelerometers can help in identifying at-risk populations and personalize treatments for healthier aging. \\

% % Höllig
% With the increasing predominance of deep learning
% methods on time series classification, interpretability becomes
% essential, especially in high-stake scenarios. Although many
% approaches to interpretability have been explored for images
% and tabular data, time series data has been mostly neglected. We
% approach the problem of interpretability by proposing TSEvo,
% a model-agnostic multiobjective evolutionary approach to time
% series counterfactuals incorporating a variety of time series
% transformation mechanisms to cope with different types and
% structures of time series.

% By ChatGPT
% Understanding the complex relationship between physical activity and biological aging is crucial for promoting healthy aging
% and mitigating age-related health risks. In this paper, we explore the use of time series regression analysis to predict biological age
% from longitudinal physical activity data. Leveraging data from the National Health and Nutrition Examination Survey (NHANES),
% we compare the performance of three biological age models: a vanilla CNN, an adaptation of a Convolutional Long short-term
% memory (ConvLSTM) to time series, and a Temporal Convolutional Network (TCN). To address this, we propose a novel approach
% utilizing counterfactual explanations to enhance the interpretability of the predictive model. By examining how changes in physical
% activity affect biological age predictions, we provide valuable insights into the causal mechanisms underlying the aging process.
% Our findings highlight the importance of incorporating interpretability into time series regression models for predicting biological
% age, thus enabling personalized interventions and health monitoring strategies. Through the integration of wearable sensors and
% deep learning techniques, our research contributes to the advancement of continuous health risk monitoring and real-time feedback,
% paving the way for improved health outcomes and quality of life across diverse populations.
% Biological age is a useful predictor of longevity and a patient's current health status. Physical activity can serve as a digital biomarker to predict a patient's biological age. By continuously monitoring a patient's health status, they can receive recommendations to improve their health. Predicting biological age from physical activity requires complex models. We are interested in discovering how a patient can adjust their physical activity to decrease their biological age and improve their health status. This mission is called "Counterfactual Explanations for Time-Series Extrinsic Regression" and has not been extensively explored to our knowledge.
% \begin{itemize}
%     \item Digital biomarkers can be used to monitor health status
%     \item Recent studies have shown that Physical Activity can be used to estimate the Biological Age
%     \item The use of time-series regression to predict biological age requires large deep-learning models.
%     \item Those models are like big black boxes that are hard to understand.
%     \item In healthcare, understanding how DL models are making their decisions is key.
%     \item For this reason, the explainability of DL models is important.
%     \item In this paper, we explain DL models used to predict BA from time-series representing the weekly PA.
%     \item To do so, we used counterfactuals.
    
% \end{itemize}

% By ChatGPT : 

Biological age serves as a vital predictor of longevity and overall health, making it a focal point in predictive medicine. Leveraging advancements in digital health, we investigate the utility of physical activity as a digital biomarker for predicting biological age and delivering tailored health recommendations. However, this endeavor is not without its challenges, as it requires the development of intricate predictive models.

In our study, we delve into uncharted territory to explore how adjustments in physical activity can influence biological age and subsequently improve health outcomes. Through rigorous analysis and innovative methodologies, we aim to provide actionable insights for individuals seeking to optimize their health through lifestyle modifications. This mission represents a pioneering effort in the field, promising to unlock new avenues for personalized healthcare optimization.

\end{abstract}

% %%Graphical abstract
% \begin{graphicalabstract}
% \includegraphics{grabs}
% \end{graphicalabstract}

% %%Research highlights
% \begin{highlights}
% \item Research highlight 1
% \item Research highlight 2
% \end{highlights}

\begin{keyword}
%% keywords here, in the form: keyword \sep keyword
Deep Learning \sep Time Series \sep Extrinsic Regression \sep Counterfactuals \sep Explanations%% PACS codes here, in the form: \PACS code \sep code
% \PACS 0000 \sep 1111
%% MSC codes here, in the form: \MSC code \sep code
%% or \MSC[2008] code \sep code (2000 is the default)
% \MSC 0000 \sep 1111
\end{keyword}

\end{frontmatter}

%% \linenumbers

%% main text

\section{Introduction}
\label{sec:introduction}

Healthcare costs are globally increasing due to an aging population, technological advancements, medication errors, and a rise in annual spending on medicines  \cite{fries_james_f_reducing_1993, bodenheimer_high_2005}. The aging population, poor diet, physical inactivity, and tobacco use (including secondhand smoke) \cite{cdc_chronic_2022} contribute to the prevalence of chronic diseases, which are a major cause of deaths among the populations \cite{huzooree_pervasive_2019}. Digital health \cite{varshney_mobile_2014} can fulfill the need for healthcare accessible to everyone, regardless of location or time, while also improving the quality of care and reducing costs. The use of portable edge devices with sensing capabilities allows for the remote monitoring of patient health data, which can be particularly helpful for those with chronic conditions \cite{javaid_sensors_2021}. With wearable sensors, mobile phones, or other edge devices, patients can easily record physiological and behavioral data, which can be aggregated to create digital biomarkers that explain, influence, or predict health-related outcomes. Passively measured data such as vital signs, physical activity, and other health-related data allows patients to monitor their health condition without visiting a healthcare provider~\cite{coravos_developing_2019}. These real-time and remote monitoring capabilities not only improve patient outcomes but can also reduce healthcare costs by minimizing the need for frequent visits with clinicians.

Processing huge amounts of sequence data typically requires versatile, high-performing, and highly generalizing \gls{dl} models.
 However, most existing studies have concentrated on tertiary prevention~\cite{barata_bitemporal_2024}, which aims to prevent disease recurrence or complications. Tertiary prevention only deals with diseases that have already occurred and does not proactively reduce the burden on healthcare systems. Therefore, it is crucial to shift the focus towards the early detection of diseases (secondary prevention) or even preventing diseases from occurring in the first place (primary prevention)~\cite{vlachopoulos_role_2015}.
Both primary and secondary prevention can be very beneficial in preventing the onset of serious health concerns. Research in this field has been limited due to uncertainty about which factors to examine. In general, it is difficult to evaluate the overall health condition of a healthy individual in the absence of disease symptoms.
% We do not yet have a method to assess the general health state of a healthy individual.
One potential method to determine the general health state of a person is by using the concept of biological age. Prior work has shown that it is possible to use deep learning models to predict a person's biological age non-invasively using physical activity data~\cite{pyrkov_extracting_2018, rahman_deep_2019}.
Nevertheless, existing methods of predicting biological age lack an explanation about what a person can do to improve their health state in general. It is widely known that general recommendations such as taking a minimum number of steps each day~\cite{tudor-locke_how_2011}, maintaining good sleeping habits~\cite{shim_wearable-based_2023}, and engaging in recreational activities~\cite{saxena_mental_2005} have a significant impact on health outcomes. However, specific recommendations tailored to the individual's needs are currently lacking, making it difficult to identify what changes to make at a personal level. In order to provide these recommendations, the individual needs to understand how the model assesses their health status.

Despite achieving great performance, \gls{ai} models are limited due to being seen as a black box, resulting in low practical use, especially in healthcare. \gls{xai} helps developers, domain experts, and users understand how \gls{dl} models work and how they make predictions \cite{loh_application_2022}. Many state-of-the-art tools for explaining \gls{dl} models rely on visually highlighting important input data areas, which is useful for developers or domain experts but hard for patients to understand. Counterfactual explanation systems \cite{byrne_counterfactuals_2019} aim to support counterfactual reasoning by modifying the input data to lead to a different prediction by the model. That way, the users of counterfactual explanation systems are provided with a fully diverse type of illustrative information that complies with the \gls{gdpr}\cite{wachter_counterfactual_2018} and are easy for humans to understand \cite{miller_explanation_2019}. There is a tremendous potential for counterfactual explanations in the mobile health setting \cite{lee_clinical_2024}. 

Yet, many of the \gls{xai} techniques predominantly deal with images or texts; time series data has attracted less interest, and the few techniques developed for time series are focused on tasks such as classification or forecasting \cite{theissler_explainable_2022}.     Especially in medical contexts, where relevant information often consists of time-dependent information, high-quality time series counterfactuals have the potential to give meaningful insights into decision processes.

With our work, we make the following contributions:
\begin{itemize}
    \item We present a novel approach for generating counterfactual explanations for time series extrinsic regression.
    \item We use our approach to adjust four counterfactual methods for time series classification to time series extrinsic regression.
    \item We compare qualitatively and quantitatively generated counterfactual explanations in a mobile health setting to estimate biological age from physical activity data.
    \item We illustrate how counterfactual explanations can be used to generate meaningful text recommendations and provide continuous health supervision, thus reducing the need for external supervision and, consequently, healthcare costs.
\end{itemize}

\begin{table}[h!t]
\caption{Code Availability}
\label{tab:introduction:code_and_website}
\centering
\renewcommand*{\arraystretch}{1.4}
\begin{tabularx}{\columnwidth}{l|X}
\hline
\textbf{Implementation} & Python, R\\
\textbf{License} & MIT \\
% \textbf{Documentation} & \url{https://www.claid.ethz.ch}\\
% \textbf{Available packages} & pip, pub, aar (maven)\\
\textbf{Code repository} & \url{https://github.com/RealLast/BA-Estimation-TCN}\\
\hline
\end{tabularx}
\end{table}

%============ Written by ChatGPT ============

\section{Related Work}
\label{sec:related-work}
% \textbf{Explainable Artificial Intelligence on Time-Series Data} An Overview of existing Explainable Artificial Intelligence (XAI) on Time-Series Data shows that the main focus is on the classification task.

% We are interested in post-hoc explainability methods, with a preference for model-agnostic types. Post-hoc methods refer to the fact that the explainability module wraps the model to produce an explanation. On the other hand, ante-hoc methods incorporate the explainability module into the model's architecture. The agnostic type refers to the fact that the explainability methods do not depend on the model and should work with any type of model. 

% The scope of the explanation can be either local or global. The local scope would mean that the methods could explain which behaviours for the patient are good or bad. General scope tends to identify good or bad behaviour in the whole population.

% The target of the explanation is the patient himself.

% And the problem type should be extrinsic regression.
% 
\begin{table*}[h!]
  \centering
  \begin{tabular}{@{}ccccccccccc@{}}
    \toprule
    Date        &   Authors                         & Model             & Methods       & Spec./Agno.   & Scope         & Target  & Problem Type                      & Citations         & Code \\
    \midrule

    2013        & Senin\cite{senin_sax-vsm_2013}             &\textit{Sax-vsm}   & SAX           & Specific      & Global        & Dev.      & Classification                    & 379               & \href{https://github.com/jMotif/sax-vsm_classic}{code}\\ 
    
    2016        &   Wang \cite{wang_time_2016}        & \textit{FCN}      & Backprop.     & Specific      & Local         & Dev.      & Classification                    & 1688              & \href{https://github.com/cauchyturing/UCR_Time_Series_Classification_Deep_Learning_Baseline}{code}\\ 

    2017        & Karim\cite{karim_multivariate_2019}             &\textit{\footnotesize{LSTM-FCN}} & Attention     &Specific       & Local         & Dev.      & Classification                    & 1143              & \href{https://github.com/houshd/LSTM-FCN}{code}\\ 

    2018        & Wachter\cite{wachter_counterfactual_2018} & \textit{W-CF} & CF    & Agnostic      & Local         & User      & Classification & 2584 & \href{https://github.com/e-delaney/Instance-Based_CFE_TSC/tree/main/W-CF}{code}\\

    2018        & Vinayavekhin\cite{vinayavekhin_focusing_2018} &\textit{TCL}& Attention &Specific        & Local         & Dev.      & Class. \& Fore.                           & 25                & no\\  

    2018        & Strodthoff\cite{strodthoff_detecting_2019}  &\textit{\footnotesize{MI-FCNN}}   & Backprop.     &Specific       & Local         & User   & Classification                    & 176               & no\\ 

    2018        & Siddiqui\cite{siddiqui_tsviz_2019}       &\textit{Tsviz}     & Backprop.     &Specific       & Both          & Dev.      & Class. \& Reg.                    & 78                & \href{https://github.com/shoaibahmed/TSViz-Core}{code}\\

    2018        & Le Nguyen\cite{nguyen_interpretable_2018}&\textit{SEQL}&SAX        &Specific       & Global        & Dev.      & Classification                    & 11                & \href{https://github.com/lnthach/Mr-SEQL}{code}\\ 

    2018        & Cho\cite{cho_interpretation_2020}   &\textit{CHAP}      & Backprop.     & Specific      & Both          & Dev.      & Classification                           & 176               & no\\

    2018        & Ge\cite{ge_interpretable_2018}      &\textit{\footnotesize{ICU-LSTM}}  & Attention     &Specific       & Global        & Dev.      & Prediction                        & 68                & no \\  

    2018        & El-Sappagh\cite{el-sappagh_ontology-based_2018}   & \textit{FAHP}     & Fuzy logic    &Specific       & Global        & User      & Classification                    & 78                & no\\ 

    2019        &   Fawaz \cite{ismail_fawaz_accurate_2019}  & \textit{\footnotesize{ESS-CNN}}  & Backprop.     &Specific       & Local         & User   & Class. \& Reg.                    & 86                & \href{https://github.com/hfawaz/ijcars19}{code}\\ 

    2019        &  Wang\cite{wang_learning_2019}      &\textit{PR-CNN} & Sapelets      &Specific       & Global        & Dev.      & Classification                    & 17                & no\\ 

    2019        &   Oviedo \cite{oviedo_fast_2019}    & \textit{AutoXRD}  & Backprop.     &Specific       & Both          & Dev.      & Classification                    & 198               & \href{https://github.com/PV-Lab/autoXRD/tree/master}{code}\\ 

    2019        & Kashiparekh\cite{kashiparekh_convtimenet_2019}&\textit{ConvNet}& Perturbation  &Agnostic       & Local         & Dev.      & Classification                    & 89                & no \\ 

    2019        & Choi\cite{choi_fully_2021}               &\textit{IDH}& Attention   &Specific       & Local         & Dev.      & Classification                                 & 92                & \href{https://github.com/yoonchoi-neuro/automated_hybrid_IDH}{code}\\  

    2019        & Tonekaboni\cite{tonekaboni_explaining_2019}&\textit{FFC}& CF&Agnostic    & Both          & Dev.      & Classification                           & 8                & no \\ 

    2019        & Assaf\cite{assaf_mtex-cnn_2019}                  &\textit{Mtex-cnn}  & Backprop.&Specific       & Local         & Dev.      & Forecasting                        & 51                & no\\ 

    2019        & Munir\cite{munir_tsxplain_2019}            &\textit{Tsxplain}  &  Backprop.    &Agnostic       & Local         & Dev.      & Classification                    & 17                & no\\  

    2019        & Gee\cite{gee_explaining_2019}        &\textit{PDL}       & Prototypes    &Specific       & Global        & Dev.      & Classification                    & 66                & \href{https://github.com/alangee/ijcai19-ts-prototypes}{code} \\ 

    2020        & Li\cite{li_efficient_2022}                   &\textit{\footnotesize{BSPCOVER}}  & Shapelets     & Specific      & Global        & Dev.      & Classification                    & 43               & no\\  

    2020        & Augustin\cite{augustin_adversarial_2020}&\textit{RATIO} & CF&Agnostic      & Local         & Dev.      & Training                          & 70                & \href{https://github.com/M4xim4l/InNOutRobustness}{code}\\ 
    
    2020        & Hao\cite{hao_new_2020}                 &\textit{CA-SFCN}   & Attention     &Specific       & Local         & Dev.      & Classification                    & 30                & \href{https://github.com/huipingcao/nmsu_yhao_ijcai2020}{code}\\ 

    2020        &   Wolanin\cite{wolanin_estimating_2020}       & \textit{CY-EDL}   & Backprop.     &Specific       & Both          & User   & Forecasting                          & 125               & no \\

    2020        & Kidger\cite{kidger_generalised_2020}& \textit{GST}      & Shapelets     &Specific       & Global        & Dev.      & Classification                    & 9                 & \href{https://github.com/patrick-kidger/generalised_shapelets}{code}\\  

    2020        & Schockaert\cite{schockaert_attention_2020} & \textit{\footnotesize{AM-LSTM}}& Attention&Specific       & Both          & Dev.      & Forecasting                       & 6                 & no\\ 

    2020        & Siddiqui\cite{siddiqui_tsinsight_2020}& \textit{Tsinsight}& Attention   &Specific       &Both           & Dev.      & Classification                    & 12                & no \\ 

    2020        & Tan\cite{tan_explainable_2021}                 &\textit{\footnotesize{eUA-CRNN}}  & Attention     &Specific       & Local         & Dev.      & Classification                                & 28                & no\\  

    2020        & Pan\cite{pan_series_2020}           &\textit{Saliency}& Perturbation    &Agnostic       & Local         & Dev.      & Forecasting                       & 7                 & no \\ 

    2020        & Ismail\cite{ismail_benchmarking_2020}           &\textit{Saliency}& Perturbation    &Agnostic       & Local         & Dev.      & Classification                    & 124               & \href{https://github.com/ayaabdelsalam91/TS-Interpretability-Benchmark}{code}\\ 

    2020        & Wang\cite{wang_deep_2021}           & \textit{\footnotesize{Deep-FCM}}& Fuzzy logic    &Specific       & Global        & Dev.      & Prediction                        & 39                & no\\ 

    2020        & Lauristsen\cite{lauritsen_early_2020}   &\textit{TCN}       &Backprop.      &Specific       & Both          & Dev.      & Classification                    & 234               & \href{https://github.com/albermax/innvestigate}{1} and \href{https://github.com/slundberg/shap}{2} \\

    2021        &Crabbé\cite{crabbe_explaining_2021}    & \textit{DynaMask}     & Perturbation  & Agnostic      & Local     & Dev.      & Class. \& Reg.                    & 55            & \href{https://github.com/JonathanCrabbe/Dynamask}{code}   \\ 

    2021        & Lim\cite{lim_temporal_2020}         & \textit{TFT}      & Attention     &Specific       & Both          & Dev.      & Forecasting                       & 822               & \href{https://github.com/greatwhiz/tft_tf2}{code}       \\ 

    2021        & Delaney\cite{delaney_instance-based_2021} & \textit{\footnotesize{Native Guide}} & CF & Agnostic & Local & Both      & Classification & 92 & \href{https://github.com/e-delaney/Instance-Based_CFE_TSC/tree/main}{code} \\
  
    2022        &Gao\cite{gao_explainable_2022}       &  \textit{ETNODE}  & Attention     &Specific       & Both          & Dev.      & Forecasting                        & 8                 & \href{https://github.com/PengleiGao/ETN-ODE}{code} \\  

    2023        &Zhao\cite{zhao_explainable_2023}                &\textit{Att-TCN}   & Perturbation             & Agnostic             & Both             & Dev.          & Classification                    & 3                 & no \\ 

    2023        & Höllig\cite{hollig_tsevo_2022}           & \textit{TSEvo}& CF           & Agnostic             & Local             & User         & Classification                                 & 5                 & \href{https://github.com/fzi-forschungszentrum-informatik/TSInterpret}{code} \\

    2024        & Ours           & \textit{TSEvoR}& CF           & Agnostic             & Local             & User         & Regression                   & -                 & \href{https://github.com/AlexisTabin/BA-Estimation-TCN}{code} \\

    \bottomrule
  \end{tabular}
  \caption{Comparison of the existing XAI techniques for Time-Series, adapted from \cite{rojat_explainable_2021}}
\end{table*}

% \textbf{Counterfactuals Explanations for Time-Series Classification} Counterfactuals are very close to Adversarial Perturbations \cite{moosavi-dezfooli_universal_2017}. The main difference between them is the sparsity factor. Adversarial Perturbations are used mainly when handling image classification. The goal is to show that a very similar image could be classified differently with changes in the input that are unnoticeable by the human eye. To achieve that, the Adversarial Perturbation model modifies many variables by a small value, whereas the Counterfactuals want to modify the least number possible of variables to achieve human interpretable solutions. Wachter \& Al. \cite{wachter_counterfactual_2018} were among the first to propose a Counterfactual theory. 
% Definition of Counterfactuals, Wachter equation 

% \textbf{GAP - From Counterfactuals for Time-Series Classification to Counterfactuals for Time-Series Extrinsic Regression}
% Explain existing techniques

%============ Written by ChatGPT ============
\subsection{Explainable Artificial Intelligence on Time-Series Data} Existing research on Explainable Artificial Intelligence (XAI) for time-series data predominantly focuses on classification tasks. \\
Here, the emphasis will be put on post-hoc explainability methods, specifically prioritizing techniques that work across different types of models, known as model-agnostic approaches.
Post-hoc methods wrap an explainability module around the model to generate explanations, contrasting with ante-hoc methods that integrate explainability within the model's architecture.
Model-agnostic methods are preferred due to their versatility across various model types. \\
Additionally, explainability scopes vary, encompassing both local and global perspectives.
Local explanations provide insights into individual behaviours, while global explanations discern broader trends within populations.
Crucially, the patients themselves are the target audience for explanations, aligning with the overarching goal of personalized healthcare interventions.\\
Furthermore, the problem type addressed typically falls under the name of time-series extrinsic regression, where the goal is to learn a relationship between a time-serie and a continuous scalar variable \cite{tan_time_2021}.


\begin{table*}[h!]
  \centering
  \begin{tabular}{@{}ccccccccccc@{}}
    \toprule
    Date        &   Authors                         & Model             & Methods       & Spec./Agno.   & Scope         & Target  & Problem Type                      & Citations         & Code \\
    \midrule

    2013        & Senin\cite{senin_sax-vsm_2013}             &\textit{Sax-vsm}   & SAX           & Specific      & Global        & Dev.      & Classification                    & 379               & \href{https://github.com/jMotif/sax-vsm_classic}{code}\\ 
    
    2016        &   Wang \cite{wang_time_2016}        & \textit{FCN}      & Backprop.     & Specific      & Local         & Dev.      & Classification                    & 1688              & \href{https://github.com/cauchyturing/UCR_Time_Series_Classification_Deep_Learning_Baseline}{code}\\ 

    2017        & Karim\cite{karim_multivariate_2019}             &\textit{\footnotesize{LSTM-FCN}} & Attention     &Specific       & Local         & Dev.      & Classification                    & 1143              & \href{https://github.com/houshd/LSTM-FCN}{code}\\ 

    2018        & Wachter\cite{wachter_counterfactual_2018} & \textit{W-CF} & CF    & Agnostic      & Local         & User      & Classification & 2584 & \href{https://github.com/e-delaney/Instance-Based_CFE_TSC/tree/main/W-CF}{code}\\

    2018        & Vinayavekhin\cite{vinayavekhin_focusing_2018} &\textit{TCL}& Attention &Specific        & Local         & Dev.      & Class. \& Fore.                           & 25                & no\\  

    2018        & Strodthoff\cite{strodthoff_detecting_2019}  &\textit{\footnotesize{MI-FCNN}}   & Backprop.     &Specific       & Local         & User   & Classification                    & 176               & no\\ 

    2018        & Siddiqui\cite{siddiqui_tsviz_2019}       &\textit{Tsviz}     & Backprop.     &Specific       & Both          & Dev.      & Class. \& Reg.                    & 78                & \href{https://github.com/shoaibahmed/TSViz-Core}{code}\\

    2018        & Le Nguyen\cite{nguyen_interpretable_2018}&\textit{SEQL}&SAX        &Specific       & Global        & Dev.      & Classification                    & 11                & \href{https://github.com/lnthach/Mr-SEQL}{code}\\ 

    2018        & Cho\cite{cho_interpretation_2020}   &\textit{CHAP}      & Backprop.     & Specific      & Both          & Dev.      & Classification                           & 176               & no\\

    2018        & Ge\cite{ge_interpretable_2018}      &\textit{\footnotesize{ICU-LSTM}}  & Attention     &Specific       & Global        & Dev.      & Prediction                        & 68                & no \\  

    2018        & El-Sappagh\cite{el-sappagh_ontology-based_2018}   & \textit{FAHP}     & Fuzy logic    &Specific       & Global        & User      & Classification                    & 78                & no\\ 

    2019        &   Fawaz \cite{ismail_fawaz_accurate_2019}  & \textit{\footnotesize{ESS-CNN}}  & Backprop.     &Specific       & Local         & User   & Class. \& Reg.                    & 86                & \href{https://github.com/hfawaz/ijcars19}{code}\\ 

    2019        &  Wang\cite{wang_learning_2019}      &\textit{PR-CNN} & Sapelets      &Specific       & Global        & Dev.      & Classification                    & 17                & no\\ 

    2019        &   Oviedo \cite{oviedo_fast_2019}    & \textit{AutoXRD}  & Backprop.     &Specific       & Both          & Dev.      & Classification                    & 198               & \href{https://github.com/PV-Lab/autoXRD/tree/master}{code}\\ 

    2019        & Kashiparekh\cite{kashiparekh_convtimenet_2019}&\textit{ConvNet}& Perturbation  &Agnostic       & Local         & Dev.      & Classification                    & 89                & no \\ 

    2019        & Choi\cite{choi_fully_2021}               &\textit{IDH}& Attention   &Specific       & Local         & Dev.      & Classification                                 & 92                & \href{https://github.com/yoonchoi-neuro/automated_hybrid_IDH}{code}\\  

    2019        & Tonekaboni\cite{tonekaboni_explaining_2019}&\textit{FFC}& CF&Agnostic    & Both          & Dev.      & Classification                           & 8                & no \\ 

    2019        & Assaf\cite{assaf_mtex-cnn_2019}                  &\textit{Mtex-cnn}  & Backprop.&Specific       & Local         & Dev.      & Forecasting                        & 51                & no\\ 

    2019        & Munir\cite{munir_tsxplain_2019}            &\textit{Tsxplain}  &  Backprop.    &Agnostic       & Local         & Dev.      & Classification                    & 17                & no\\  

    2019        & Gee\cite{gee_explaining_2019}        &\textit{PDL}       & Prototypes    &Specific       & Global        & Dev.      & Classification                    & 66                & \href{https://github.com/alangee/ijcai19-ts-prototypes}{code} \\ 

    2020        & Li\cite{li_efficient_2022}                   &\textit{\footnotesize{BSPCOVER}}  & Shapelets     & Specific      & Global        & Dev.      & Classification                    & 43               & no\\  

    2020        & Augustin\cite{augustin_adversarial_2020}&\textit{RATIO} & CF&Agnostic      & Local         & Dev.      & Training                          & 70                & \href{https://github.com/M4xim4l/InNOutRobustness}{code}\\ 
    
    2020        & Hao\cite{hao_new_2020}                 &\textit{CA-SFCN}   & Attention     &Specific       & Local         & Dev.      & Classification                    & 30                & \href{https://github.com/huipingcao/nmsu_yhao_ijcai2020}{code}\\ 

    2020        &   Wolanin\cite{wolanin_estimating_2020}       & \textit{CY-EDL}   & Backprop.     &Specific       & Both          & User   & Forecasting                          & 125               & no \\

    2020        & Kidger\cite{kidger_generalised_2020}& \textit{GST}      & Shapelets     &Specific       & Global        & Dev.      & Classification                    & 9                 & \href{https://github.com/patrick-kidger/generalised_shapelets}{code}\\  

    2020        & Schockaert\cite{schockaert_attention_2020} & \textit{\footnotesize{AM-LSTM}}& Attention&Specific       & Both          & Dev.      & Forecasting                       & 6                 & no\\ 

    2020        & Siddiqui\cite{siddiqui_tsinsight_2020}& \textit{Tsinsight}& Attention   &Specific       &Both           & Dev.      & Classification                    & 12                & no \\ 

    2020        & Tan\cite{tan_explainable_2021}                 &\textit{\footnotesize{eUA-CRNN}}  & Attention     &Specific       & Local         & Dev.      & Classification                                & 28                & no\\  

    2020        & Pan\cite{pan_series_2020}           &\textit{Saliency}& Perturbation    &Agnostic       & Local         & Dev.      & Forecasting                       & 7                 & no \\ 

    2020        & Ismail\cite{ismail_benchmarking_2020}           &\textit{Saliency}& Perturbation    &Agnostic       & Local         & Dev.      & Classification                    & 124               & \href{https://github.com/ayaabdelsalam91/TS-Interpretability-Benchmark}{code}\\ 

    2020        & Wang\cite{wang_deep_2021}           & \textit{\footnotesize{Deep-FCM}}& Fuzzy logic    &Specific       & Global        & Dev.      & Prediction                        & 39                & no\\ 

    2020        & Lauristsen\cite{lauritsen_early_2020}   &\textit{TCN}       &Backprop.      &Specific       & Both          & Dev.      & Classification                    & 234               & \href{https://github.com/albermax/innvestigate}{1} and \href{https://github.com/slundberg/shap}{2} \\

    2021        &Crabbé\cite{crabbe_explaining_2021}    & \textit{DynaMask}     & Perturbation  & Agnostic      & Local     & Dev.      & Class. \& Reg.                    & 55            & \href{https://github.com/JonathanCrabbe/Dynamask}{code}   \\ 

    2021        & Lim\cite{lim_temporal_2020}         & \textit{TFT}      & Attention     &Specific       & Both          & Dev.      & Forecasting                       & 822               & \href{https://github.com/greatwhiz/tft_tf2}{code}       \\ 

    2021        & Delaney\cite{delaney_instance-based_2021} & \textit{\footnotesize{Native Guide}} & CF & Agnostic & Local & Both      & Classification & 92 & \href{https://github.com/e-delaney/Instance-Based_CFE_TSC/tree/main}{code} \\
  
    2022        &Gao\cite{gao_explainable_2022}       &  \textit{ETNODE}  & Attention     &Specific       & Both          & Dev.      & Forecasting                        & 8                 & \href{https://github.com/PengleiGao/ETN-ODE}{code} \\  

    2023        &Zhao\cite{zhao_explainable_2023}                &\textit{Att-TCN}   & Perturbation             & Agnostic             & Both             & Dev.          & Classification                    & 3                 & no \\ 

    2023        & Höllig\cite{hollig_tsevo_2022}           & \textit{TSEvo}& CF           & Agnostic             & Local             & User         & Classification                                 & 5                 & \href{https://github.com/fzi-forschungszentrum-informatik/TSInterpret}{code} \\

    2024        & Ours           & \textit{TSEvoR}& CF           & Agnostic             & Local             & User         & Regression                   & -                 & \href{https://github.com/AlexisTabin/BA-Estimation-TCN}{code} \\

    \bottomrule
  \end{tabular}
  \caption{Comparison of the existing XAI techniques for Time-Series, adapted from \cite{rojat_explainable_2021}}
\end{table*}

\subsection{Counterfactual Explanations for Time-Series Classification} Counterfactual explanations offer a compelling approach, differing from Adversarial Perturbations primarily in their emphasis on sparse modifications for human interpretability.
While Adversarial Perturbations focus on imperceptible changes to input data for image classification tasks, counterfactual methods strive to alter the fewest variables necessary to achieve interpretable solutions. \\


\subsubsection{Wachter} Pioneering work by Wachter et al. \cite{wachter_counterfactual_2018} laid the foundation for counterfactual theory, offering clear definitions and methodologies, including the influential Wachter equation.

%============ From Wachter paper ============
\begin{equation} \label{eq:1}
\arg \min _{w} \ell\left(f_{w}\left(x_{i}\right), y_{i}\right)+\rho(w)
\end{equation}

Where $y_{i}$ is the label for data point $x_{i}$ and $\rho(\cdot)$ is a regularizer over the weights. We wish to find a counterfactual $x^{\prime}$ as close to the original point $x_{i}$ as possible such that $f_{w}\left(x^{\prime}\right)$ is equal to a new target $y^{\prime}$. We can find $x^{\prime}$ by holding $w$ fixed and minimizing the related objective.


\begin{equation} \label{eq:2}
\arg \min _{x^{\prime}} \max _{\lambda} \lambda\left(f_{w}\left(x^{\prime}\right)-y^{\prime}\right)^{2}+d\left(x_{i}, x^{\prime}\right)
\end{equation}

Where $d(\cdot, \cdot)$ is a distance function that measures how far the counterfactual $x^{\prime}$ and the original data point $x_{i}$ are from one another. In practice, maximisation over $\lambda$ is done by iteratively solving for $x^{\prime}$ and increasing $\lambda$ until a sufficiently close solution is found.

The choice of optimiser for these problems is relatively unimportant. In practice, any optimiser capable of training the classifier under Equation 1 seems to work equally well, and we use $\mathrm{ADAM}$ \cite{kingma_adam_2017} for all experiments.
As local minima are a concern, we initialise each run with different random values for $x^{\prime}$ and select as our counterfactual the best minimizer of Equation 2. These different minima can be used as a diverse set of multiple counterfactuals.
%============ From Wachter paper ============
\subsubsection{Instance-Based}
% \documentclass[10pt]{article}
% \usepackage[utf8]{inputenc}
% \usepackage[T1]{fontenc}
% \usepackage{amsmath}
% \usepackage{amsfonts}
% \usepackage{amssymb}
% \usepackage[version=4]{mhchem}
% \usepackage{stmaryrd}
% \usepackage{bbold}
% \usepackage{graphicx}
% \usepackage[export]{adjustbox}
% \graphicspath{ {./images/} }
% \usepackage{hyperref}
% \hypersetup{colorlinks=true, linkcolor=blue, filecolor=magenta, urlcolor=cyan,}
% \urlstyle{same}

%New command to display footnote whose markers will always be hidden
\let\svthefootnote\thefootnote
\newcommand\blfootnotetext[1]{%
  \let\thefootnote\relax\footnote{#1}%
  \addtocounter{footnote}{-1}%
  \let\thefootnote\svthefootnote%
}

%Overriding the \footnotetext command to hide the marker if its value is `0`
\let\svfootnotetext\footnotetext
\renewcommand\footnotetext[2][?]{%
  \if\relax#1\relax%
    \ifnum\value{footnote}=0\blfootnotetext{#2}\else\svfootnotetext{#2}\fi%
  \else%
    \if?#1\ifnum\value{footnote}=0\blfootnotetext{#2}\else\svfootnotetext{#2}\fi%
    \else\svfootnotetext[#1]{#2}\fi%
  \fi
}

Like other case-based XAI methods $25,27,31,41$, at its core Native Guide relies upon existing instances in the training data, so-called native guides or nearest unlike neighbors (NUNs), that it retrieves and adapts to generate counterfactual explanations (see Figure 3). In this section, we outline the two main steps in the algorithm, after first describing the notation adopted.

\textbf{Notation.} Staying consistent with the notation of 15,18], a time series $T=\left\{\left\langle t_{1}, t_{2}, \ldots, t_{m}\right\rangle\right\}$ is an ordered set of real values, where $m$ is the length. A time series data set $\mathbf{T}=\left\{T_{1}, T_{2} \ldots, T_{n}\right\} \in \mathbb{R}^{n \times m}$ is a collection of such time series where each time series has a class label $c$ forming a vector of class labels $\mathbf{Y} \in \mathbb{Z}$. Consider a black-box classifier $b(T)$ that takes a time series $T$ as an input and predicts a probability output $P(\mathbf{Y} \mid T)$ over the label output space. Given a to-be-explained query time series $T_{q}$, with predicted label $c$ from the black-box classifier (formally $b\left(T_{q}\right)=c$ ), a counterfactual explanation aims to find how $T_{q}$ needs to change for the system to classify it alternatively, as $c^{\prime}$. We refer to $T^{\prime}$ as a counterfactual explanation for $T_{q}$ such that $b\left(T^{\prime}\right)=c^{\prime}$. Although there are many candidate solutions for $T^{\prime}$, the method prioritizes those that meet the four key properties of proximity, sparsity, plausibility and diversity.

\begin{figure}[h]
    \centering
    \includegraphics[width=0.4\textwidth]{CF/images/instance-based.jpg}
    \caption{A query time series $T_{q}$ (X with solid arrow) and a nearest-unlike neighbor, $T_{N U N}^{\prime}$ (red circle with solid arrow) are used to guide the generation of counterfactual $T^{\prime}$ (see yellow circle) in a binary classification task. Another in-sample counterfactual (i.e., the next NUN; other red circle with dashed arrow) could also be used to generate another counterfactual for diverse explanations.}
\end{figure}


\textbf{Step 1:} Retrieve native guide. Given a query time series, $T_{q}$, find a counterfactual instance, $T_{\text {Native }}^{\prime}$, that exists in the case-base. An example of one such instance is the query's nearest unlike neighbor $\left(T_{N U N}^{\prime}\right)$. In using these "native counterfactual" cases the method guarantees the explanation's plausibility as it is, by definition, within the distribution. However, such instances are not guaranteed to be sufficiently proximate to the query or, indeed, sparse, so an adaption step is necessary to generate the "explanatory counterfactual", $T^{\prime}$ (see Figure $3)$.

\textbf{Step 2:} Adapt native guide to generate counterfactual. To produce a more proximate explanatory counterfactual, $T^{\prime}$, the native guide, $T_{\text {Native }}^{\prime}$ is perturbed towards the to-be-explained query-case, $T_{q}$ (see Figure 3). Typically, counterfactual methods use some $L_{p}$ distance metric to guide this perturbation (such as Manhattan distance, [55]) and in time series where dynamic time warping (DTW) distance is often more appropriate an analogous averaging technique known as weighted dynamic barycentre averaging can be used 13]. In cases where we are explaining a deep-learner's predictions, the feature-weight vectors of the classifier, $\boldsymbol{\omega}$, can be used to perturb "semantically-meaningful" features of the time series, rather than the "raw" time series data, to guarantee sparsity ${ }^{4}$.
\footnotetext{Note, SHAP can also be used to generate such vectors, if we are directly explaining any given model, rather than twinning.
}

Accordingly, using the feature-weights, the method seeks to modify contiguous, subsequences, rather than the whole time series, as follows:

$$
\begin{aligned}
& T_{q}=\left\{<t_{1}, t_{2}, t_{3}, t_{4}, t_{5} \ldots, t_{n}>\right\} \text { s.t. } b\left(T_{q}\right)=c \\
& T^{\prime}=\left\{<t_{1}, t_{2}^{\prime}, t_{3}^{\prime}, t_{4}^{\prime}, t_{5} \ldots, t_{n}>\right\} \text { s.t. } b\left(T^{\prime}\right)=c^{\prime}
\end{aligned}
$$



\footnotetext{\href{https://github.com/e-delaney/Instance-Based_CFE_TSC}{https://github.com/e-delaney/Instance-Based\_CFE\_TSC}
}



%============ From TSEvo paper ============
\subsubsection{TSEvo}
We study a supervised time series classification problem. Let $x=\left[x_{1}, \ldots, x_{T}\right] \in \mathbb{R}^{N \times T}$ be a uni- or multivariate time series, where $T$ is the number of time steps and $N$ is the number of features. Let $x_{i, t}$ be the input feature $i$ at time $t$. Similarly, let $X_{:, t} \in R^{N}$ and $X_{i,:} \in R^{T}$ be the feature vector at time $t$, and the time vector for feature $i$, respectively. $Y$ denotes the output, and $f: x \rightarrow Y$ a classification model returning a probability distribution vector over classes $Y=\left[y_{1}, \ldots, y_{C}\right]$, where $C$ is the total number of classes (i.e., outputs) and $y_{i}$ the probability of $x$ belonging to class $i$. The classification model $f$ is seen as a "black-box" - i.e., no access to the inner workings of a model are available. Only the result $Y$ is observable.

\begin{definition}
Goal Properties of the TSEvo counterfactual search.
$$
\begin{aligned}
& \mathbf{R}_{\mathbf{1}}: \min \left(d\left(x, x^{c f}\right)\right), \text { s.t. } f(x) \neq f\left(x^{c f}\right) \\
& \mathbf{R}_{\mathbf{2}}: \min \left(\sum_{i=1}^{N} \sum_{t=1}^{T} \mathbbm{1}_{\left|x_{i, t}-x_{i, t}^{c f}\right| !=0}\right), \text { s.t. } f(x) \neq f\left(x^{c f}\right) \\
& \\
& \mathbf{R}_{\mathbf{3}}: x^{c f} \sim D, \text { s.t. } f(x) \neq f\left(x^{c f}\right)
\end{aligned}
$$ 
\end{definition}
The goal of counterfactual approaches is, given a time series $x$ and a classifier $f$, to provide an explanation via counter examples allowing humans to understand why classifier $f$ chose class $y$ for data point $x$ and not a counterfactual class $y^{c f}$ \cite{wachter_counterfactual_2018}. To allow such understanding, we assume that for each $x$, a counterfactual sample $x^{c f}$ can be computed, that is close to $x$, but differently classified $y \neq y^{c f}$. The resulting $x^{c f}$ is supposed to be a proximate $\left(\mathbf{R}_{1}\right)$ \cite{mothilal_explaining_2020}, sparse $\left(\mathbf{R}_{\mathbf{2}}\right)$ \cite{mothilal_explaining_2020}, and plausible $\left(\mathbf{R}_{3}\right)$ \cite{laugel_dangers_2019} adaption of $x$. Proximity refers to the distance between the query instance $x$ and the counterfactual instance $x_{c f}$, calculated as a distance measure $d$ between $x$ and $x^{c f}$. Sparsity refers to the number of feature changes between $x$ and $x_{c f}$. A plausible adaption indicates that the resulting $x^{c f}$ is in distribution with the available data $D$.

Combining the desired properties $\mathbf{R}_{1}$ and $\mathbf{R}_{\mathbf{2}}$, with a function for guiding the output distance away from the original classification leads to multi-objective problem $O$. Equation 1 shows the minimization problem. $O_{1}$ is derived from $\mathbf{R}_{1}$ by applying Mean Absolute Error as distance function $d$ \cite{mothilal_explaining_2020}, \cite{wachter_counterfactual_2018}. $O_{2}$ is consistent with $\mathbf{R}_{2}$ and $O_{3}$ denotes the output distance maximizing the output distance on a target class $l$. If no target class is chosen the second highest class probability is designated as the target.

\begin{multline}
\min O(x):=\left(O_{1}\left(x, x^{c f}\right), O_{2}\left(x, x^{c f}\right), O_{3}\left(x^{c f}\right)\right) \\
\text { s.t.f }(x) \neq f\left(x^{c f}\right) \\
O_{1}\left(x, x^{c f}\right)=\frac{1}{N * T} \sum_{i=1}^{N} \sum_{t=1}^{T}\left|x_{i, t}-x_{i, t}^{c f}\right| \\
O_{2}\left(x, x^{c f}\right)=\frac{1}{N * T} \sum_{i=1}^{N} \sum_{t=1}^{T} \mathbbm{1}_{\left|x_{i, t}-x_{i, t}^{c f}\right| \neq 0} \\
O_{3}\left(x^{c f}\right)=1-f\left(x^{c f}\right)_{l}    \\
\end{multline}

TODO : Explain how TSEvo then find the best CF

%============ From TSEvo paper ============

\subsection{Gap - From Counterfactuals for Time-Series Classification to Counterfactuals for Time-Series Extrinsic Regression} Despite advancements in counterfactual explanations for time-series classification, there exists a notable gap in extending these techniques to address extrinsic regression tasks within time-series data analysis. Bridging this gap is essential for enabling the interpretation of AI-derived insights in the context of individual health trajectories, thereby facilitating actionable recommendations tailored to individual needs.
%============ Written by ChatGPT ============


%% The Appendices part is started with the command \appendix;
%% appendix sections are then done as normal sections

% \input{table}

\section{Methods}
\label{sec:methods}
\textbf{Time-Series Extrinsic Regression}
\begin{definition}
Let $x=\left[x_{1}, \ldots, x_{T}\right] \in \mathbb{R}^{N \times T}$ be a uni- or multivariate time series, where $T$ is the number of time steps, and $N$ is the number of features.
Let $x_{i, t}$ be the input feature $i$ at time $t$.\\
$Y$ denotes the output, and $f: x \rightarrow Y$ a regression model returning a extrinsic continuous variable. The regression model $f$ is seen as a "black box" - i.e., no access to the inner workings of a model is available. Only the result $Y$ is observable.    
\end{definition}

\textbf{Counterfactual search} The goal of counterfactual approaches is, given a time series $x$ and a classifier $f$, to provide an explanation via counter-examples, allowing humans to understand why model $f$ predicts $y$ for data point $x$ and not a counterfactual class $y^{c f}$ \cite{wachter_counterfactual_2018}. To allow such understanding, we assume that for each $x$, a counterfactual sample $x^{c f}$ can be computed that is close to $x$ but the difference between their prediction is larger than a certain threshold $|y - y^{c f}| > \epsilon$. The resulting $x^{c f}$ is supposed to be a proximate $\left(\mathbf{R}_{1}\right)$ \cite{mothilal_explaining_2020}, sparse $\left(\mathbf{R}_{\mathbf{2}}\right)$ \cite{mothilal_explaining_2020}, and plausible $\left(\mathbf{R}_{3}\right)$ \cite{laugel_dangers_2019} adaption of $x$. Proximity refers to the distance between the query instance $x$ and the counterfactual instance $x_{c f}$, calculated as a distance measure $d$ between $x$ and $x^{c f}$. Sparsity refers to the number of feature changes between $x$ and $x_{c f}$. A plausible adaption indicates that the resulting $x^{c f}$ is in distribution with the available data $D$. \\

\textbf{Note:} In our specific case, as we want to find a healthier patient, the predicted value for the counterfactual has to be lower than the observation's prediction $y - y^{c f} > \epsilon$

\subsection{Adapting TSEvo for Classification to Extrinsic Regression}
To achieve this we adapted the TSEvo algorithms so it can find counterfactuals in an extrinsic regression problem.
The adapted objectives are defined below :


\begin{definition}
Desired Properties of the counterfactual.
$$
\begin{aligned}
& \mathbf{R}_{\mathbf{1}}: \min \left(d\left(x, x^{c f}\right)\right), \text { s.t. } f(x) - f\left(x^{c f}\right) > \epsilon\\
& \mathbf{R}_{\mathbf{2}}: \min \left(\sum_{i=1}^{N} \sum_{t=1}^{T} \mathbbm{1}_{\left|x_{i, t}-x_{i, t}^{c f}\right| !=0}\right), \text { s.t. } f(x) - f\left(x^{c f}\right) > \epsilon\\
& \\
& \mathbf{R}_{\mathbf{3}}: x^{c f} \sim D, \text { s.t. } f(x) - f\left(x^{c f}\right) > \epsilon
\end{aligned}
$$ 
\end{definition}

Combining the desired properties $\mathbf{R}_{1}$ and $\mathbf{R}_{\mathbf{2}}$, with a function for guiding the output distance away from the original classification leads to multi-objective problem $O$. Equation 1 shows the minimization problem. $O_{1}$ is derived from $\mathbf{R}_{1}$ by applying Mean Absolute Error as distance function $d$ \cite{mothilal_explaining_2020}, \cite{wachter_counterfactual_2018}. $O_{2}$ is consistent with $\mathbf{R}_{2}$ and $O_{3}$ denotes the output distance maximizing the output distance on a target class $l$. If no target class is chosen the second highest class probability is designated as the target.

\begin{multline}
\min O(x):=\left(O_{1}\left(x, x^{c f}\right), O_{2}\left(x, x^{c f}\right), O_{3}\left(x^{c f}\right)\right) \\
\text { s.t. } f(x) - f\left(x^{c f}\right) > \epsilon \\
O_{1}\left(x, x^{c f}\right)=\frac{1}{N * T} \sum_{i=1}^{N} \sum_{t=1}^{T}\left|x_{i, t}-x_{i, t}^{c f}\right| \\
O_{2}\left(x, x^{c f}\right)=\frac{1}{N * T} \sum_{i=1}^{N} \sum_{t=1}^{T} \mathbbm{1}_{\left|x_{i, t}-x_{i, t}^{c f}\right| \neq 0} \\
O_{3}\left(x^{c f}\right)=\left(f(x) - f\left(x^{c f}\right)-\epsilon\right) / \epsilon    \\
\end{multline}

The multiobjective optimisation follows the same steps as the one described in the TSEvo paper \cite{hollig_tsevo_2022}.
In summary, a population of $n$ individuals is initialized. For $g$ generation, the evolution algorithms will select the best individuals in the population, perform cross-over with a certain probability and/or mutate them with a certain probability. The individuals are evaluated with respect to their objectives score. 
To mutate and generate plausible counterfactuals $(R3)$, a reference set $R$ is initialized. This reference set $R$ is necessary for the mutation. The reference set $R = {z \in D : f(x)-\epsilon \geq f(z) \geq f(x)-2*\epsilon}$ is a subset of all known data $D$ with a prediction other than the original class $f(x)$.

\section{Results}
\label{sec:results}
\begin{figure*}[h!]
     \centering
     % \textbf{\rotatebox{90}{CNN$\,\;$}}
     \begin{subfigure}[b]{0.24\textwidth}
         \centering
         \includegraphics[width=\textwidth]{images/6306/0_6306_TCN_Wachter_cf.pdf}
         \caption{\gls{wachter} Counterfactual}
         \label{fig:cf:wachter}
     \end{subfigure}
     \hfill
     \begin{subfigure}[b]{0.24\textwidth}
         \centering
         \includegraphics[width=\textwidth]{images/6306/4_6306_TCN_NUN_cf.pdf}
         \caption{NUNR Counterfactual}
         \label{fig:cf:nun}
     \end{subfigure}
     \hfill
     \begin{subfigure}[b]{0.24\textwidth}
         \centering
         \includegraphics[width=\textwidth]{images/6306/1_6306_TCN_DBA_cf.pdf}
         \caption{DBAR Counterfactual}
         \label{fig:cf:dba}
     \end{subfigure}
    \hfill
     \begin{subfigure}[b]{0.24\textwidth}
         \centering
         \includegraphics[width=\textwidth]{images/6306/2_6306_TCN_TSEvo_cf.pdf}
         \caption{TSEvoR Counterfactual}
         \label{fig:cf:tsevo}
     \end{subfigure}

          \begin{subfigure}[b]{0.24\textwidth}
         \centering
         \includegraphics[width=\textwidth]{images/6306/0_6306_TCN_Wachter_reco.pdf}
         \caption{\gls{wachter} Recommendations}
         \label{fig:reco:wachter}
     \end{subfigure}
     \hfill
     \begin{subfigure}[b]{0.24\textwidth}
         \centering
         \includegraphics[width=\textwidth]{images/6306/4_6306_TCN_NUN_reco.pdf}
         \caption{NUNR Recommendations}
         \label{fig:reco:nun}
     \end{subfigure}
     \hfill
     \begin{subfigure}[b]{0.24\textwidth}
         \centering
         \includegraphics[width=\textwidth]{images/6306/1_6306_TCN_DBA_reco.pdf}
         \caption{DBAR Recommendations}
         \label{fig:reco:dba}
     \end{subfigure}
    \hfill
     \begin{subfigure}[b]{0.24\textwidth}
         \centering
         \includegraphics[width=\textwidth]{images/6306/2_6306_TCN_TSEvo_reco.pdf}
         \caption{TSEvoR Recommendations}
         \label{fig:reco:tsevo}
     \end{subfigure}

    \caption{Example of time series counterfactuals (pink line) of an input time
series (blue line) for the biological estimation problem, where, given a threshold $\varepsilon=3$, the counterfactual modifies the original activity so the DL model predicts a smaller biological age. Each plot represents a different counterfactual technique, and each corresponding text recommendation is listed below the plot.}
    \label{fig:quali-eval}
\end{figure*}

In the results section, we first report the performance of the \gls{dl} models to estimate Biological Age data~(cf.~Section~\ref{sec:results:dl-training}). Then, we evaluate the generated \gls{cfe} using qualitative and quantitative criteria~(cf.~Section~\ref{sec:results:cf}).

\subsection{Biological Age Estimation with Deep Learning}
\label{sec:results:dl-training}
We set out to reproduce results presented in prior work \cite{pyrkov_extracting_2018, rahman_deep_2019} that use \gls{dl} models to estimate biological age from physical activity data and achieved the following results~(cf.~Table~\ref{tab:models}).
In Rahman et al. work \cite{rahman_deep_2019}, the authors reported slightly different results. For the \gls{convlstm}, they reported a \gls{mae} of $13.21$ years, an \gls{mse} of $282,58$ with a Pearson correlation of $0.62$. For the \gls{cnn}, they reported an \gls{mae} of $15.49$, an \gls{mse} of $353.82$, and a Pearson correlation of $0.45$.
\begin{table}[h!]
    \centering
    \begin{tabular}{cccc}
        \toprule
        \textbf{Model} & \textbf{\gls{mse}$\,{}^\downarrow$} & \textbf{\gls{mae}$\,{}^\downarrow$} & \makecell{\textbf{Pearson} \\\textbf{Correlation}$\,{}^\uparrow$}\\
        \midrule
        \textbf{\gls{tcn}} & 367.69 & 14.85 & 0.49\\
        \textbf{\gls{cnn}} & 501.33 & 17.50 & 0.23\\
        \textbf{\gls{convlstm}} & 488.25 & 17.35 & 0.22\\
        \bottomrule
    \end{tabular}
    \caption{Comparison of the models' performances}
    \label{tab:models}
\end{table}

\subsection{Counterfactuals for Biological Age Estimation}
\label{sec:results:cf}
In this section, we evaluate the generated counterfactuals in two ways. First, qualitatively, we plotted one patient evaluated with the four different counterfactual techniques. The plots allow us to evaluate the user interpretability of the explanations. Looking at the explanations, a user should understand what he does well, what he could improve, and what impact it will have on his health. Then, we evaluate quantitatively using metrics based on the properties defined in Section \ref{sec:methods}. 

\subsubsection{Qualitative evaluation}
 Fig. \ref{fig:quali-eval} shows counterfactuals obtained for a specific patient using the four counterfactual methods. We highlight findings for each method in the following subsections.

\subsubsection{\gls{wachter}}
Figure~\ref{fig:cf:wachter} shows the counterfactual result for patient 6306. It was obtained using the \gls{tcn} model and the \gls{wachter} technique. The model's predicted biological labels are at the plot's top. According to the \gls{tcn} model, the patient's biological age is 46.46.
% \textbf{Contrastiveness} From the plot and the recommendations, it is not possible for the user to understand what changes are necessary to improve his biological age.\\
% \textbf{Selectivity} Again, the user cannot understand the explanations and the recommendations, which should be his main priority.\\
% \textbf{Social} \gls{wachter}'s changes are invisible on the plot. The user needs precise physical activity control to apply these changes; thus, the explanations are unrealistic. \\
% \textbf{Truthful} The counterfactual's level of physical activity is the same as the user's, which makes it plausible. \\
% \textbf{Consistent with prior beliefs} From the plot, the explanation suggests that having almost the same physical activity results in a drop of twelve years for the predicted biological age (!), which is not consistent with prior beliefs.\\
The counterfactual physical activity level corresponds to that of someone who is 34.1 years old, which is not a valid counterfactual, as valid counterfactuals need to have a biological age between 40.46 and 43.46 years old. The \gls{wachter} technique modifies slightly the physical activity recorded at every timestamp to find a counterfactual, resulting in hardly interpretable explanations. The explanations could not give any recommendations.

\subsubsection{\gls{nunr}}
Figure~\ref{fig:cf:nun} displays the counterfactual outcome for patient $6306$ using the \gls{tcn} model explained with the \gls{nunr} technique.
% \textbf{Contrastiveness} The \gls{nunr} technique requires a change in the patient's everyday routine. With the recommendations, the focus is on Sunday and Wednesday mornings and Friday afternoons. \\
% \textbf{Selectivity} The whole physical activity is highlighted, not only a subset. The text recommendations improve selectivity by giving the top three timestamps that should be changed.\\
% \textbf{Social} Based on the plot and the recommendations, the user has to change his habits most of the day of the week.\\
% \textbf{Truthful} Concretely, the \gls{nunr}-CF shows the physical activity of someone at least three years younger. So, as someone is doing this activity, the activity is plausible.\\
% \textbf{Consistent with prior beliefs} Recommendations mostly suggest that the user should decrease his physical activity, but it is a common belief that doing more physical activity is healthier overall.\\
The \gls{nunr} biological age is 42.39, which is a valid result. Based on the explanation, four recommendations were made to improve the patient's health; only the top three are shown in the plot. These include reducing activity levels on Wednesday and Sunday mornings and Saturday afternoons and increasing activity on Friday afternoons.

\subsubsection{\gls{dbar}}
Figure~\ref{fig:cf:dba} shows the counterfactual for patient $6306$ obtained through the \gls{tcn} model with the \gls{dbar} technique.
% \textbf{Contrastiveness} The plot only does not indicate necessary changes. When adding the recommendations, the focus on Friday becomes apparent.\\
% \textbf{Selectivity} Similar to \gls{nunr}, many data points are affected by the explanations, and again, the recommendation highlights the most important subset of the physical activity.\\
% \textbf{Social} Based on the plot and the recommendations, the user has to change his habits most of the day of the week. \\
% \textbf{Truthful} The DBA average introduces spikes in the physical activity.\\
% \textbf{Consistent with prior beliefs} The DBA averaging removed the main inconsistency in NUN recommendations, as the counterfactual no longer suggests decreasing physical activity. \\
The \gls{dbar} biological age is 43.46, which is an optimal and valid result. It is optimal because 43.46 is the closest accepted label possible. By comparing with the \gls{nunr} plot, we can observe that the \gls{dbar} technique averages between the \gls{nunr}'s physical activity and the patient's physical activity to produce a similar counterfactual, with less important changes (percentage changes are lower) but still slight changes at each time stamp. Based on the \gls{dbar}'s output, the recommender suggests only one recommendation to improve the patient's biological age: increasing physical activity on Friday afternoon. It is important to note that this was already a recommendation from the \gls{nunr} technique.

\subsubsection{\gls{tsevor}}
Figure~\ref{fig:cf:tsevo} shows a counterfactual for patient $6306$, which was obtained using the \gls{tcn} model and explained using the \gls{tsevor} technique.
% \textbf{Contrastiveness} The plot highlights a large area on Friday and three smaller areas on Monday, Tuesday, and Wednesday. This is then narrowed to only Friday by the text recommendation.\\
% \textbf{Selectivity} From the plot, the user observes three main changes: Monday morning, Tuesday morning, and Friday midday. The recommendation helps the user select Friday as the main area of focus.\\
% \textbf{Social} The recommendations are realistic as they focus mainly on one day. \\
% \textbf{Truthful} The counterfactual's physical activity is plausible. \\
% \textbf{Consistent with prior beliefs} The explanations suggest increasing physical activity on a day when the user had very little activity. \\
The \gls{tsevor} biological age is 43.46, which is an optimal and valid result. \gls{tsevor} only had to make a few changes to the timestamps to arrive at a valid and optimal result. Based on this analysis, it then recommended that the patient increase their physical activity on Friday morning and afternoon, which is the same as \gls{dbar} and \gls{nunr}.


\subsection{Quantitative evaluation}
\begin{table*}[h!]
    \large
    \caption{Objectives}
    \centering
    \resizebox{0.96\textwidth}{!}{
        \small
        \begin{tabular}{@{ }l@{\hspace{4mm}}l@{\hspace{7mm}}c@{\hspace{10mm}}l@{\hspace{10mm}}l@{\hspace{10mm}}l@{\hspace{10mm}}c@{\hspace{10mm}}}
                \toprule[1pt]%    
                && \textbf{Validity}$\,{}^\uparrow$& \textbf{Proximity}$\,{}^\downarrow$& \textbf{Sparsity}$\,{}^\downarrow$& \textbf{Plausibility}$\,{}^\uparrow$& \textbf{Time}$\,{}^\downarrow$
                    \\
                    \midrule%
                \multirow{4}{*}{\textbf{\rotatebox{90}{CNN$\,\;$}}}\n& \textbf{TSEvoR}& 0.37±0.48& 159.99±274.71& \textbf{0.04±0.09}& 0.39±0.25& 8m01s±02m52s\\& \textbf{NUNR}& \textbf{0.99±0.08}& 4740.66±1149.04& 0.98±0.02& \textbf{0.47±0.20}& \textbf{0m01s±00m01s}\\& \textbf{DBAR}& 0.10±0.31& 2499.74±702.91& 0.99±0.03& 0.02±0.11& 9m28s±04m58s\\& \textbf{\gls{wachter}}& 0.00±0.00& \textbf{0.00±0.00}& 1.00±0.00& 0.00±0.00& 1m01s±00m28s\\\n\bottomrule[1pt]\multirow{4}{*}{\textbf{\rotatebox{90}{ConvLSTM$\,\;$}}}\n& \textbf{TSEvoR}& 0.08±0.27& 450.75±445.8& \textbf{0.07±0.10}& 0.46±0.29& 47m36s±11m21s\\& \textbf{NUNR}&\textbf{0.99±0.08}& 3944.03±796.31& 0.98±0.02& \textbf{0.59±0.25}& \textbf{00m01s±00m01s}\\& \textbf{DBAR}& 0.09±0.29& 1898.02±328.64& 1.00±0.02& 0.04±0.15& 11m33s±04m32s\\& \textbf{\gls{wachter}}& 0.02±0.12& \textbf{0.00±0.00}& 0.99±0.02& 0.14±0.27& 06m32s±01m48s\\\n\bottomrule[1pt]\multirow{4}{*}{\textbf{\rotatebox{90}{TCN$\,\;$}}}\n& \textbf{TSEvoR}& 0.39±0.49& 147.37±242.41& \textbf{0.04±0.09}& 0.41±0.26& 10m01s±01m14s\\& \textbf{NUNR}& \textbf{0.99±0.08}& 4740.66±1149.04& 0.98±0.02& \textbf{0.51±0.23}& \textbf{00m01s±00m01s}\\& \textbf{DBAR}& 0.10±0.30& 2344.37±639.02& 1.00±0.02& 0.01±0.07& 12m03s±05m54s\\& \textbf{\gls{wachter}}& 0.00±0.00& \textbf{0.00±0.00}& 1.00±0.00& 0.00±0.00& 01m09s±00m03s\\
            \bottomrule[1pt]
            \end{tabular}
    }
    \label{tab:experiments:results}
    \vspace{-5mm}
\end{table*}



% \begin{table*}[!t]

%                 \caption{Objectives}

%                 \centering

%                 \resizebox{0.96\textwidth}{!}{%


%                 \begin{tabular}{@{ }l@{\hspace{2mm}}l@{\hspace{7mm}}c@{\hspace{1.5mm}}c@{\hspace{1mm}}c@{\hspace{7mm}}c@{\hspace{1.5mm}}c@{\hspace{1mm}}c@{\hspace{7mm}}c@{\hspace{1.5mm}}c@{\hspace{1mm}}c@{\hspace{7mm}}c@{\hspace{1.5mm}}c@{\hspace{1mm}}c@{\hspace{7mm}}c@{\hspace{1.5mm}}c@{\hspace{1mm}}c@{}}

%                 \toprule[1pt]%

%                 && \multicolumn{3}{c@{\hspace{10mm}}}{\textbf{Validity}$\,{}^\uparrow$}& \multicolumn{3}{c@{\hspace{10mm}}}{\textbf{Proximity}$\,{}^\downarrow$}& \multicolumn{3}{c@{\hspace{10mm}}}{\textbf{Sparsity}$\,{}^\downarrow$}& \multicolumn{3}{c@{\hspace{10mm}}}{\textbf{Plausibility}$\,{}^\uparrow$}& \multicolumn{3}{c@{\hspace{10mm}}}{\textbf{Time}$\,{}^\downarrow$}\\ 
% [1mm]
% && \textbf{CNN}& \textbf{ConvLSTM}& \textbf{TCN}& \textbf{CNN}& \textbf{ConvLSTM}& \textbf{TCN}& \textbf{CNN}& \textbf{ConvLSTM}& \textbf{TCN}& \textbf{CNN}& \textbf{ConvLSTM}& \textbf{TCN}& \textbf{CNN}& \textbf{ConvLSTM}& \textbf{TCN}\\ 
% \midrule%
% \multirow{4}{*}{\textbf{\rotatebox{90}{Techniques$\,\;$}}}
% & TSEvoR 	& 0.37 	& 0.08 	& 0.39 	& 159.99 	& 450.75 	& \textbf{147.37} 	& \textbf{0.04} 	& 0.07 	& \textbf{0.04} 	& 0.39 	& 0.46 	& 0.41 	& 08m 01s 	& 47m 36s 	& 10m 01s \\
% & NUNR 	& \textbf{0.99} 	& \textbf{0.99} 	& \textbf{0.99} 	& 4740.66 	& 3944.03 	& 4740.66 	& 0.98 	& 0.98 	& 0.98 	& 0.47 	& \textbf{0.59} 	& 0.51 	& \textbf{00m 00s} 	& \textbf{00m 00s} 	& \textbf{00m 00s} \\
% & DBAR 	& 0.1 	& 0.09 	& 0.1 	& 2499.74 	& 1898.02 	& 2344.37 	& 0.99 	& 1.0 	& 1.0 	& 0.02 	& 0.04 	& 0.01 	& 12m 28s 	& 11m 33s 	& 12m 03s \\
% & \gls{wachter} 	& 0.0 	& 0.0 	& 0.0 	& - 	& - 	& - 	& 1.0 	& 1.0 	& 1.0 	& 0.0 	& 0.0 	& 0.0 	& 01m 01s 	& 07m 13s 	& 01m 19s \\

%                 \bottomrule[1pt]

%                 \end{tabular}

%                 }

%                 \label{tab:experiments:results}

%                 \vspace{-5mm}

%                 \end{table*}


% \begin{table*}[!t]

%     \caption{Objectives}

%                 \centering

%                 \resizebox{0.96\textwidth}{!}{%


%                 \begin{tabular}{@{ }l@{\hspace{2mm}}l@{\hspace{7mm}}c@{\hspace{1.5mm}}c@{\hspace{1mm}}c@{\hspace{7mm}}c@{\hspace{1.5mm}}c@{\hspace{1mm}}c@{\hspace{7mm}}c@{\hspace{1.5mm}}c@{\hspace{1mm}}c@{\hspace{7mm}}c@{\hspace{1.5mm}}c@{\hspace{1mm}}c@{\hspace{7mm}}c@{\hspace{1.5mm}}c@{\hspace{1mm}}c@{}}

%                 \toprule[1pt]%

%                 && \multicolumn{3}{c@{\hspace{10mm}}}{\textbf{Validity}$\,{}^\uparrow$}& \multicolumn{3}{c@{\hspace{10mm}}}{\textbf{Proximity}$\,{}^\downarrow$}& \multicolumn{3}{c@{\hspace{10mm}}}{\textbf{Sparsity}$\,{}^\downarrow$}& \multicolumn{3}{c@{\hspace{10mm}}}{\textbf{Plausibility}$\,{}^\uparrow$}& \multicolumn{3}{c@{\hspace{10mm}}}{\textbf{Time}$\,{}^\downarrow$}\\ 
% [1mm]
% && \textbf{CNN}& \textbf{ConvLSTM}& \textbf{TCN}& \textbf{CNN}& \textbf{ConvLSTM}& \textbf{TCN}& \textbf{CNN}& \textbf{ConvLSTM}& \textbf{TCN}& \textbf{CNN}& \textbf{ConvLSTM}& \textbf{TCN}& \textbf{CNN}& \textbf{ConvLSTM}& \textbf{TCN}\\ 
% \midrule%
% \multirow{4}{*}{\textbf{\rotatebox{90}{Techniques$\,\;$}}}
% & TSEvoR 	& 0.37±0.48 	& 0.08±0.27 	& 0.39±0.49 	& 159.99±274.71 	& 450.75±445.8 	& 147.37±242.41 	& 0.04±0.09 	& 0.07±0.1 	& 0.04±0.09 	& 0.39±0.25 	& 0.46±0.29 	& 0.41±0.26 	& 08m 01s±02m 52s 	& 47m 36s±11m 21s 	& 10m 01s±01m 14s \\
% & NUNR 	& 0.99±0.08 	& 0.99±0.08 	& 0.99±0.08 	& 4740.66±1149.04 	& 3944.03±796.31 	& 4740.66±1149.04 	& 0.98±0.02 	& 0.98±0.02 	& 0.98±0.02 	& 0.47±0.2 	& 0.59±0.25 	& 0.51±0.23 	& 00m 00s±00m 00s 	& 00m 00s±00m 00s 	& 00m 00s±00m 00s \\
% & DBAR 	& 0.1±0.31 	& 0.09±0.29 	& 0.1±0.3 	& 2499.74±702.91 	& 1898.02±328.64 	& 2344.37±639.02 	& 0.99±0.03 	& 1.0±0.02 	& 1.0±0.02 	& 0.02±0.11 	& 0.04±0.15 	& 0.01±0.07 	& 12m 28s±04m 58s 	& 11m 33s±04m 32s 	& 12m 03s±04m 54s \\
% & \gls{wachter} 	& 0.0±0.0 	& 0.0±0.0 	& 0.0±0.0 	& -±- 	& -±- 	& -±- 	& 1.0±0.0 	& 1.0±0.0 	& 1.0±0.0 	& 0.0±0.0 	& 0.0±0.0 	& 0.0±0.0 	& 01m 01s±00m 27s 	& 07m 13s±00m 42s 	& 01m 19s±00m 03s \\

%                 \bottomrule[1pt]

%                 \end{tabular}

%                 }

%                 \label{tab:experiments:results}

%                 \vspace{-5mm}

% \end{table*}


We generated counterfactuals on the 723 samples from the test set. To generate counterfactuals, we first feed each sample from the test set to each of the three deep-learning models (\gls{cnn}, \gls{tcn}, and \gls{convlstm}). We then generated counterfactuals using the four different counterfactual techniques (\gls{wachter}, \gls{nunr}, \gls{dbar}, and \gls{tsevor}). Finally, we evaluate the $723\times3\times4=8676$ generated counterfactuals with the following metrics:
\begin{itemize}
    \item \textbf{Validity}: Proportion of the 723 generated counterfactuals that are valid~(cf.~Def.~\ref{def:validity}). 
    \item \textbf{Proximity}: $L_1$ norm (Manhattan distance) between the query $x$ and the counterfactual $x^{cf}$.
    \begin{equation}
        ||x - x^{cf}||_1 = \sum_i|x_i - x_i^{cf}|
    \end{equation}
    \item \textbf{Sparsity}: Proportion of changed data points to obtain the counterfactual. 
    \item \textbf{Plausibility}: Proportion of Nearest Neighbours with a label close to the counterfactual label. We used $k=5$ \gls{nn}, and $\varepsilon=3$ as the threshold value to differentiate between close and far neighbors.
    \begin{equation}        \text{Plausibility}=\frac{1}{k}\sum_{i\, \in\, kNN(x^{cf})}\mathbbm{1}_{|y_i - y^{cf}|\, \leq\, \varepsilon}
    \end{equation}
    \item \textbf{Time}: Time needed to explain a sample.
\end{itemize}

Upon analysis, we can first observe that the best-performing explainable technique for each metric remains constant across the models, which only impacts the time aspect. Secondly, we note that the \gls{nunr} technique performs best across three out of five metrics but falls short on the proximity and sparsity metrics. Compared to \gls{nunr}, \gls{dbar} improves the proximity metric at the cost of a large drop in validity and plausibility. Thirdly, it is evident that all techniques, except for \gls{tsevor}, cannot produce sparse counterfactuals. It is also important to note that \gls{tsevor} ranks first or second in every metric except for time, where it secures third place. Moreover, when using \gls{tsevor}, the model plays a role in the validity and time metrics. Indeed, the validity drops to 0.08 when explaining the \gls{convlstm} model, compared to 0.37 for \gls{cnn} and 0.39 for \gls{tcn}, while being five times slower.

\section{Discussion}
\label{sec:discussion}
\textbf{Setting thresholds for regression models is brittle ?}\\
Spooner \& Al. \cite{spooner_counterfactual_2021} argue that the choice of the threshold has an influence on the found counterfactual.

From the paper itself : TODO

For scalar regression problems, one solution to specifying validity of CFs is to set a threshold $\varepsilon$, so that the $\mathrm{CF}$ set is $\mathcal{S} \doteq$ $\{x \in \mathcal{X}:|f(x)-f(q)| \geq \varepsilon\}$. This construction yields an equivalency between instances of CFX-EXISTENCE for (binary) classifiers and regressors. They are also tractable, since the targets are defined using preorders, which admit trivial verification circuits. However, thresholding does not distinguish between $\mathrm{CFs}$ in $\mathcal{S}$, which is a problem when the distance in $x$ far exceeds $\varepsilon$, leading to unrealistic CFs that are far from the query point.

\begin{figure}[h]
    \centering
    \includegraphics[width=0.4\textwidth]{images/potential.jpg}
    \caption{Threshold robustness issue with CFEs: choosing $\varepsilon_{2}$ over $\varepsilon_{1}$ yields $x_{2}$ rather than $x_{1}$ as the counterfactual.}
\end{figure}

 Furthermore, CFs defined via thresholding can be very sensitive to $\varepsilon$. Figure 1 shows an example where the query instance $q \in \mathcal{X}$ is bounded above by $q<x_{1}<x_{2}$, and $f$ is monotone increasing such that $f(q)<f(q)+\varepsilon_{1}<f\left(x_{1}\right)<f(q)+\varepsilon_{2}<f\left(x_{2}\right)$ for $0<\varepsilon_{1}<\varepsilon_{2}$. Define $\Delta \doteq x_{2}-x_{1}$. Setting $\varepsilon_{1}$ as the threshold yields the CF $x_{1}$, whereas choosing $\varepsilon_{2}$ yields $x_{2}$ instead. Considering the distance When $\Delta$ is large, $x_{2}$ is further from $q$ than $x_{1}$ is, and thus $\varepsilon_{2}$ is arguably a worse threshold than $\varepsilon_{1}$. as it yields CFs far from the query point. However, there is no ex ante way to choose between $\varepsilon_{1}$ and $\varepsilon_{2}$, which causes this threshold robustness issue. A key contribution our work is to formalise the notion of regression counterfactuals in terms of potentials instead of the direct instantiation of the primal-dual spaces via thresholds, which we will now describe.

\subsection*{Potential-Based Search}
In this work, we focus on a subset of $\mathbb{T}_{q}^{f}$ where, for a given query $q \in \mathcal{X}$, we ascribe to each output $y \in \mathcal{Y}$ a scalar potential which quantifies the value associated with candidate counterfactual points. This concept is closely related to the construction used in potential games [37] to analyse equilibria when agents' incentives are dictated by a single global function. Definition 2.3 below formalises this.
\begin{definition}
(Potential). A counterfactual potential function is an element $\rho$ of the set

$$
\mathcal{R}_{q}^{f} \doteq\left\{\rho \in C_{L}^{1}(\mathcal{Y}, \mathbb{R}): \max _{x \in \mathcal{X}} \rho(f(x))-\rho(f(q))>0\right\}
$$

of all continuously differentiable maps between model outputs and the real line, where $(f, q)$ is a model-query pair, $\rho$ and $\nabla \rho$ are $L$-Lipschitz, and the value of $\rho$ is not maximized at $q$.
\end{definition}

\section{Outlook}
\label{sec:outlook}
Some Outlook
\section*{Acknowledgements}
I want to acknowledge the invaluable contributions of several individuals who made this work possible. Firstly, I would like to express my deep appreciation for the exceptional guidance provided by Patrick and our dedicated weekly meetings. I am also grateful to Prof. Vogt for her valuable time and supervision, as well as for providing access to the very useful infrastructure of the Medical Data Science Group. I must also mention Margaux, whose wonderful support, motivation, patience, and fantastic advice were instrumental throughout my thesis journey. Her unwavering support kept me going. Additionally, I am grateful to my whole family for their reviews, encouragement, and inspiration.

% 
%% The Appendices part is started with the command \appendix;
%% appendix sections are then done as normal sections
\appendix

\section{Sample Appendix Section}
\label{sec:sample:appendix}
Lorem ipsum dolor sit amet, consectetur adipiscing elit, sed do eiusmod tempor section \ref{sec:sample1} incididunt ut labore et dolore magna aliqua. Ut enim ad minim veniam, quis nostrud exercitation ullamco laboris nisi ut aliquip ex ea commodo consequat. Duis aute irure dolor in reprehenderit in voluptate velit esse cillum dolore eu fugiat nulla pariatur. Excepteur sint occaecat cupidatat non proident, sunt in culpa qui officia deserunt mollit anim id est laborum.

%% If you have bibdatabase file and want bibtex to generate the
%% bibitems, please use
%%
\newpage
\printglossaries
\newpage
\bibliographystyle{elsarticle-harv} 
\bibliography{references}

%% else use the following coding to input the bibitems directly in the
%% TeX file.

% \begin{thebibliography}{00}

% %% \bibitem[Author(year)]{label}
% %% Text of bibliographic item

% \bibitem[ ()]{}
\clearpage

% \end{thebibliography}
\end{document}

\endinput
%%
%% End of file `elsarticle-template-harv.tex'.
