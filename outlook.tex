\section{Outlook}
\label{sec:outlook}
% In our exploration, each step of the pipeline~(cf.~Fig.~\ref{fig:pipeline}) reveals opportunities for refinement and expansion:
% \begin{enumerate}

%  \item \textit{Data:} Our experiment used physical activity data from 2003 to 2006. However, there is potential to improve our dataset by including data available until 2014. Moreover, we could enrich our analysis by incorporating diverse digital biomarkers such as heart rate and tension into our analysis, using multivariate time series instead of univariate. By doing so, we can enhance the predictive power of our model.
% % We also acknowledge the importance of data ethics, privacy, and usage in mobile health. It is crucial to gain patients' trust in the data collection process for the pervasive collection of digital biomarkers. Therefore, we must adopt transparent communication and ethical practices to ensure patient privacy and data ethics.
%  % \item \textit{\gls{tser}:} Our study highlights the importance of conducting more research in the \gls{tser} field to enhance model accuracy. We have identified a gap in estimating biological age from physical activity, where the best model achieved an \gls{mae} of 12 years. We suggest exploring alternative methodologies or incorporating novel features to improve model performance and obtain more precise estimations.
%  \item \textit{\gls{tsxair}}: Based on recent research by Letzgus et al. \cite{letzgus_toward_2022} and Shim et al. \cite{shim_wearable-based_2023}, we can improve our explainable framework by using Contextualised \gls{xai} techniques and population clusters based on physical activity patterns. We aim to enhance our approach by exploring alternative mutation techniques suggested in the \gls{tsevo} paper and using \gls{llm}s or text-based models for automated recommendations. These improvements will empower our approach and make it more effective.
%  \item \textit{Recommendations:} Analyzing the extensive archive of TSER datasets compiled by Guijo et al. \cite{guijo-rubio_unsupervised_2023} provides an opportunity to gain insights and potentially incorporate expert recommendations into our recommender system. Additionally, it is essential to test our method with diverse datasets, particularly those with well-established outcomes, to ensure that our counterfactual explanations align with common medical understanding. This validation step is crucial in enhancing the credibility and applicability of our approach in real-world scenarios.
% \end{enumerate}
% Considering these suggestions, we aim to refine our methodology and advance the field of mobile health by fostering trust, improving accuracy, and enhancing the interpretability of our model's recommendations.

As mentioned in the previous section, our approach can be applied to any \gls{dl} technique to predict a continuous variable from time series data. In future work, we would like to improve our methods and recommendation pipeline further pipeline (cf. Fig.~\ref{fig:pipeline}). We will focus on three main areas for improvement.
\textbf{Data} Our current experiment uses physical activity data from 2003 to 2006. To improve our dataset, we plan to include data available until 2014. Additionally, we will enrich our analysis by incorporating diverse digital biomarkers such as heart rate and tension, shifting from univariate to multivariate time series. These enhancements are expected to improve the predictive power of our model, thereby enabling a deeper understanding of the four methods in various scenarios. \textbf{\gls{tsxair}} Building on recent research by Letzgus et al. \cite{letzgus_toward_2022} and Shim et al. \cite{shim_wearable-based_2023}, we aim to enhance our \gls{xai} framework. We will incorporate contextualized \gls{xai} techniques and population clusters based on physical activity patterns, enhancing the explanation's interpretability.
Furthermore, we will explore alternative mutation techniques suggested in the \gls{tsevo} paper and leverage \gls{llm} or text-based models for automated recommendations. These improvements could make our approach more robust and effective. \textbf{Recommendations} We plan to improve our recommendation system by analyzing the extensive \gls{tser} datasets compiled by Guijo et al. \cite{guijo-rubio_unsupervised_2023}, which will provide valuable insights and enable us to integrate expert recommendations. Additionally, we will test our method with diverse datasets, especially those with well-established outcomes, to ensure that our counterfactual explanations align with common medical understanding. This validation step is crucial for enhancing the credibility and applicability of our approach in real-world scenarios.
By considering these improvements, we aim to refine our methodology and advance the field of mobile health. We aim to foster trust, improve accuracy, and enhance the interpretability of our model's recommendations.